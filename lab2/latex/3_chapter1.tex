\chapter{Вспомогательные функции и библиотеки}
\label{ch:chap1}

\definecolor{codegreen}{rgb}{0,0.6,0}
\definecolor{codegray}{rgb}{0.5,0.5,0.5}
\definecolor{codepurple}{rgb}{0.58,0,0.82}
\definecolor{backcolour}{rgb}{0.95,0.95,0.92}

\lstdefinestyle{mystyle}{
    backgroundcolor=\color{backcolour},   
    commentstyle=\color{codegreen},
    keywordstyle=\color{magenta},
    numberstyle=\tiny\color{codegray},
    stringstyle=\color{codepurple},
    basicstyle=\ttfamily\footnotesize,
    breakatwhitespace=false,         
    breaklines=true,                 
    captionpos=b,                    
    keepspaces=true,                 
    numbers=left,                    
    numbersep=5pt,                  
    showspaces=false,                
    showstringspaces=false,
    showtabs=false,                  
    tabsize=2
}

\lstset{style=mystyle}

Онлайн версию кода в colab можно глянуть \href{https://colab.research.google.com/drive/18OXQjzdn8iWHcQShWfElPcsidweJTsYt?usp=sharing}{здесь}. 
Учтите, что аккорды не загружены, поэтому конвертировать и загружать их придётся самостоятельно :(
\section{Задание 1}
Были использованы следующие библиотеки:
\lstinputlisting[language=Python]{listings/listing0.py}
Кусочно-заданные функции реализовывал через функционал numpy, на примере первой функции прямоугольника:
\lstinputlisting[language=Python]{listings/listing1.py}
Пример кода для построения графиков в matplotlib, больше можно увидеть в онлайн-блокноте:
\lstinputlisting[language=Python]{listings/listing2.py}
Реализовывал проверку равенства Парсеваля через интегрирование на Scipy, которое не всегда хорошо работает, это стоит учитывать:
\lstinputlisting[language=Python]{listings/listing3.py}

\section{Задание 2}

Проверка равенства Парсеваля осталась та же, как в первом задании.

Функция сдвинутого треугольника была реализована по аналогии с прошлым заданием:
\lstinputlisting[language=Python]{listings/listing4.py}

\section{Задание 3}

Для чтения звуковой волны с файла .mp3 с помощью библиотеки librosa:
\lstinputlisting[language=Python]{listings/listing5.py}

Для построения графика волны выше: 
\lstinputlisting[language=Python]{listings/listing6.py}

Дополнительные функции для численного интегрирования "ручками" образа Фурье для волны:
\lstinputlisting[language=Python]{listings/listing7.py}


\endinput