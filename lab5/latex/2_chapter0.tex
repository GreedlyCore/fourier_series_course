\chapter{Исходный код}
\label{ch:chap0}

\definecolor{codegreen}{rgb}{0,0.6,0}
\definecolor{codegray}{rgb}{0.5,0.5,0.5}
\definecolor{codepurple}{rgb}{0.58,0,0.82}
\definecolor{backcolour}{rgb}{0.95,0.95,0.92}

\lstdefinestyle{mystyle}{
    backgroundcolor=\color{backcolour},   
    commentstyle=\color{codegreen},
    keywordstyle=\color{magenta},
    numberstyle=\tiny\color{codegray},
    stringstyle=\color{codepurple},
    basicstyle=\ttfamily\footnotesize,
    breakatwhitespace=false,         
    breaklines=true,                 
    captionpos=b,                    
    keepspaces=true,                 
    numbers=left,                    
    numbersep=5pt,                  
    showspaces=false,                
    showstringspaces=false,
    showtabs=false,                  
    tabsize=2
}

\lstset{style=mystyle}

Онлайн версию кода здесь нет, делал вычисления в live-script матлабовских, в \href{https://github.com/GreedlyCore/fourier_series_course}{репозитории} можно найти исходники. 

Основные фишки кода были помечены сразу там в комментариях и много красивого латеха, поэтому прошу всех желающих заглянуть под капот непосредственно\dots 

\endinput