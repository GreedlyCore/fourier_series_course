\chapter{Размытие изображения}
\label{ch:chap2}

\definecolor{codegreen}{rgb}{0,0.6,0}
\definecolor{codegray}{rgb}{0.6,0.6,0.6}
\definecolor{codepurple}{rgb}{0.68,0,0.82}
\definecolor{backcolour}{rgb}{0.95,0.95,0.92}

\lstdefinestyle{mystyle}{
    backgroundcolor=\color{backcolour},   
    commentstyle=\color{codegreen},
    keywordstyle=\color{magenta},
    numberstyle=\tiny\color{codegray},
    stringstyle=\color{codepurple},
    basicstyle=\ttfamily\footnotesize,
    breakatwhitespace=false,         
    breaklines=true,                 
    captionpos=b,                    
    keepspaces=true,                 
    numbers=left,                    
    numbersep=5pt,                  
    showspaces=false,                
    showstringspaces=false,
    showtabs=false,                  
    tabsize=2
}
\lstset{style=mystyle}

\begin{figure}[ht]
    \centering
    \includegraphics[width=0.5\textwidth]{mirele.png}
	\caption{Оригинал}
\end{figure}

Будем делать размытие по среднему и по Гауссу, для этого подготовим несколько матриц:

Три матрицы блочного размытия, для каждой из которой мы выбираем размерность $n \geq 3$, некоторые примеры таких матриц:
$$
\begin{aligned}
    (n=3)\rightarrow\begin{pmatrix}
        \frac{1}{9} & \frac{1}{9} & \frac{1}{9}\\
        \frac{1}{9} & \frac{1}{9} & \frac{1}{9}\\
        \frac{1}{9} & \frac{1}{9} & \frac{1}{9} 
    \end{pmatrix}
    (n=4)\rightarrow\begin{pmatrix}
        \frac{1}{16} & \frac{1}{16} & \frac{1}{16} & \frac{1}{16} \\
        \frac{1}{16} & \frac{1}{16} & \frac{1}{16} & \frac{1}{16} \\
        \frac{1}{16} & \frac{1}{16} & \frac{1}{16} & \frac{1}{16} \\
        \frac{1}{16} & \frac{1}{16} & \frac{1}{16} & \frac{1}{16} 
    \end{pmatrix}
    \rightarrow\dots
\end{aligned}
$$

Три матрицы размытия по Гауссу, для каждой из которой мы выбираем размерность $n \geq 3$, и формируем внутренности за счёт следующей функции:
$$
f(x, y) = e^{-\frac{9}{n^2}( (x - \frac{n+1}{2})^2 + (y - \frac{n+1}{2})^2 )}
$$
После эту матрицу нужно пронормировать(=поделить) на сумму всех элементов внутри неё. Матрица в итоге станет центрально-симметрична. Не стану приводить численные примеры, их можно увидеть в \texttt{MATLAB} коде.
\newpage
\section{Через свёртку напрямую}
Делаем свёртку исходного изображения с каждым из ядер для каждого из размытий, получаем следующие результаты:

\begin{figure}[ht]
    \centering
    \includegraphics[width=0.6\textwidth]{mirele_MED1.png}
	\caption{Размытие по среднему, n = 5}
\end{figure}

\begin{figure}[ht]
    \centering
    \includegraphics[width=0.6\textwidth]{mirele_GAUSSIAN1.png}
	\caption{Размытие по Гауссу, n = 5}
\end{figure}
\newpage
\begin{figure}[ht]
    \centering
    \includegraphics[width=0.6\textwidth]{mirele_MED2.png}
	\caption{Размытие по среднему, n = 13}
\end{figure}

\begin{figure}[ht]
    \centering
    \includegraphics[width=0.6\textwidth]{mirele_GAUSSIAN2.png}
	\caption{Размытие по Гауссу, n = 13}
\end{figure}
\newpage
\begin{figure}[ht]
    \centering
    \includegraphics[width=0.6\textwidth]{mirele_MED3.png}
	\caption{Размытие по среднему, n = 19}
\end{figure}

\begin{figure}[ht]
    \centering
    \includegraphics[width=0.6\textwidth]{mirele_GAUSSIAN3.png}
	\caption{Размытие по Гауссу, n = 19}
\end{figure}
\newpage
Почему-то только при эффекте размытия наложение трех каналов после применение эффекта даёт не только размытое изображение, но и урезанное-убитое в цветовом плане - сравните оригинал и любое другое полученное... Я проверил код, и он абсолютно идентичен для ядер далее, но далее изображения не искажаются так,
то есть как будто бы проблема именно в применении эффекта, но пока что я корень зла не нашёл, однако размытие есть, поэтому работаем с тем, что имею. В плане степени эффекта - при $n>10$ эффект более чем заметен.

\newpage
\section{Через использование теоремы о свёртке}
Теорема о свёртке, которую мы использовали в прошлых лабах, с небольшими модификациями, может быть применима и сейчас:
$$
X \ast Y = (\hat{X}) \cdot (\hat{Y})
$$
В качестве модификации нам придется дополнить незначительными нулями размерность матрицы ядра размытия до размера изображения. В моём случае оно квадратное, чтобы немного упростить себе жизнь.

Будем искать Фурье-образ от исходного изображения и от каждого из ядер, заполнив пропуски нулями. После перемножаем их, и делаем обратное преобразование, чтобы посмотреть как сработало применение фильтра. Ниже можно взглянуть как через теорему о свёртке у нас это получилось:

\begin{figure}[ht]
    \centering
    \includegraphics[width=0.6\textwidth]{mirele2_MED1.png}
	\caption{Размытие по среднему, n = 5}
\end{figure}

\begin{figure}[ht]
    \centering
    \includegraphics[width=0.6\textwidth]{mirele2_GAUSSIAN1.png}
	\caption{Размытие по Гауссу, n = 5}
\end{figure}
\newpage
\begin{figure}[ht]
    \centering
    \includegraphics[width=0.6\textwidth]{mirele2_MED2.png}
	\caption{Размытие по среднему, n = 13}
\end{figure}
\newpage
\begin{figure}[ht]
    \centering
    \includegraphics[width=0.6\textwidth]{mirele2_GAUSSIAN2.png}
	\caption{Размытие по Гауссу, n = 13}
\end{figure}

\begin{figure}[ht]
    \centering
    \includegraphics[width=0.6\textwidth]{mirele2_MED3.png}
	\caption{Размытие по среднему, n = 19}
\end{figure}
\newpage
\begin{figure}[ht]
    \centering
    \includegraphics[width=0.6\textwidth]{mirele2_GAUSSIAN3.png}
	\caption{Размытие по Гауссу, n = 19}
\end{figure}


\section{Мини-выводы}
Мы могли заметить невооружённым взглядом, что теорема о свёртке работает и в случае её $2D$ применения, 
ведь результаты совпадают - перемножая образы, или делая свёртку непосредственно.

Касательно качества блочного и гауссовского размытия вроде бы ничего содержательного не скажешь, кроме того, как по среднему действует жёстко и однообразно, 
а по Гауссу - мягче и приятнее для взгляда, что неудивительно ведь она использует внутри себя "что-то" про нормальное распределение.
\endinput